\documentclass{article}
\usepackage{graphicx} % required for inserting images
\usepackage{amsmath,amsfonts,amssymb,amsthm,mathtools} % math
\usepackage[unicode, pdftex]{hyperref} % hyperlinks
\usepackage{amsfonts} % NZQRC

%  Русский язык
\usepackage[T2A]{fontenc}			% кодировка
\usepackage[utf8]{inputenc}			% кодировка исходного текста
\usepackage[english,russian]{babel}	% локализация и переносы

\begin{document}

\section{Топология}
\subsection{Предел последовательности точек в n–мерном евклидовом пространстве}

\subsection{Теорема Больцано–Вейерштрасса и критерий Коши сходимости последовательности}
\subsection{Внутренние, предельные, изолированные точки множества}
идея: любая окрестность принадлежит А \\
$x \in X$ - внутренняя точка множества $A \subset X$ $\Leftrightarrow$
\begin{equation*}
    \exists \epsilon > 0: U_\epsilon (x) \subset A
\end{equation*}
идея: посл из А стрем к х \\
$x \in X$ - предельная точка множества $A$ относительно X $\Leftrightarrow$
\begin{equation*}
    \exists \{ x_k \} \subset A: \underset{n \rightarrow \infty}{lim} x_k = x, \; x_n \neq x \; \forall n \in \mathbb N
\end{equation*}
идея: любая прокол окр точки из А пустая относительно А \\
$x \in X$ - изолированная точка множества $A  \in X$ $\Leftrightarrow$
\begin{equation*}
    x \in A \; and \; \exists \delta > 0 : \overset{o}{U}_\delta (x) \cap A = \emptyset
\end{equation*}
\subsection{Открытые и замкнутые множества, их свойства}
идея: внутренние точки \\
Открытое множество  $\Leftrightarrow$
\begin{equation*}
    int A = A
\end{equation*}
идея: точки прикосновения \\
Замкнутое множество $\Leftrightarrow$
\begin{equation*}
    \overline A = A
\end{equation*}
Задача 4. $A$ -- открыто  $\Leftrightarrow$ $R^n \setminus A$ -- замкнуто. \\
Задача 1. a. $A_i$ -- открыто $\forall i \in \overline{1, n}, \; n \in \mathbb N $ $\Rightarrow$
\begin{equation*}
    int \cap A = \cap A
\end{equation*}
b. $A_i$ -- открыто $\forall i \in \mathbb N $ $\Rightarrow$
\begin{equation*}
    int \cup A = \cup A
\end{equation*}
Наоборот с замкнутыми множествами по формуле из задачи 4.
\subsection{Внутренность, замыкание и граница множества}
идея: любая окрестность принадлежит А \\
Внутренность $\Leftrightarrow$
\begin{equation*}
    int A = \{ x \in X: \exists \epsilon > 0: U_{\epsilon} (x) \subset A \}
\end{equation*}
идея: любая окрестность непустая относительно А \\
Замыкание $\Leftrightarrow$
\begin{equation*}
    \overline{A} = \{ x \in X: \forall \epsilon>0 \hookrightarrow A \cap U_\epsilon (x) = \emptyset \}
\end{equation*}
идея: замыкание без внутренности \\
Граница $\Leftrightarrow$
\begin{equation*}
    \partial A = \overline{A} \setminus int A
\end{equation*}
\subsection{Компакты}
идея: к любому числу из любой посл можно выбрать подпосл к этому числу \\
Метрическое пространство X компактно $\Leftrightarrow$
\begin{equation*}
    \forall \{ x_k \} \subset X \; and \; \forall x \in X \; \exists \{ x_{k_n} \}: \{ x_{k_n} \} \rightarrow x
\end{equation*}
идея: \\
Теорема 2. А -- компакт в метрическом пространстве Х $\Rightarrow$ A ограничено и замкнуто на Х. Обратное неверно.

\section{Непрерывность}
\subsection{Предел числовой функции нескольких переменных}
\subsection{Предел функции по множеству}
\subsection{Непрерывность функции нескольких переменных в точке и по множеству}
\subsection{Свойства функций, непрерывных на компакте — ограниченность, достижение точных нижней и верхней граней, равномерная непрерывность (теорема Кантора)}
\subsection{Теорема о промежуточных значениях функции, непрерывной в области}
\subsection{Связные множества}


\section{Дифференцируемость}
\subsection{Частные производные функции нескольких переменных}
\subsection{Дифференцируемость функции в точке, дифференциал}
\subsection{Необходимые условия дифференцируемости, достаточные условия дифференцируемости функции нескольких переменных}
\subsection{Дифференцируемость сложной функции}
\subsection{Инвариантность формы дифференциала относительно замены переменных}
\subsection{Производная по направлению и градиент, их связь и геометрический смысл}



\section{Формула Тейлора}
\subsection{Частные производные высших порядков}
\begin{equation*}
    \frac{\partial^2 f}{\partial x^j \partial x^i}(x_0) \Leftrightarrow
\end{equation*}
\begin{equation*}
    \exists U_\delta (x_0) \subset X_{\frac{\partial}{\partial x^i} f(x)} \; and \; \frac{\partial^2 f}{\partial x^j \partial x^i}(x_0) = \frac{\partial}{\partial x^j} \frac{\partial}{\partial x^i} f(x_0)
\end{equation*}
\subsection{Независимость смешанной частной производной от порядка дифференцирования}
идея:  \\
\begin{equation*}
    w(t) = f(x_0+t, y_0+t) + f(x_0, y_0) - f(x_0+t, y_0) - f(x_0, y_0+t)
\end{equation*}
\begin{equation*}
    \phi (x, t) = f(x, y_0 + t) - f(x, y_0)
\end{equation*}
\begin{equation*}
    \psi(y) = \frac{\partial f}{\partial x} (x_0+\theta_1t, y)
\end{equation*}
Теорема 1:
\begin{equation*}
    \exists U_{\delta 1} (x_0, y_0) \subset X_{\frac{\partial}{\partial x \partial y} f(x,y)} \; and \; \exists U_{\delta 2} (x_0, y_0) \subset X_{\frac{\partial}{\partial y \partial x} f(x_0, y_0)} \; and \underset{x,y \rightarrow x_0,y_0}{lim} \frac{\partial}{\partial x \partial y} f(x,y) = \frac{\partial}{\partial x \partial y} f(x_0, y_0) \;
\end{equation*}
\begin{equation*}
    and \; \underset{x \rightarrow x_0}{lim} \frac{\partial}{\partial y \partial x} f(x, y) = \frac{\partial}{\partial y \partial x} f(x_0, y_0) \Rightarrow
\end{equation*}
\begin{equation*}
    \frac{\partial}{\partial y \partial x} f(x_0, y_0) = \frac{\partial}{\partial x \partial y} f(x_0, y_0)
\end{equation*}
\subsection{Дифференциалы высших порядков, отсутствие инвариантности их формы относительно замены переменных}
Все ч.пр. до $(k-1)$ включительно опр. в окр. и дифф. в т.$x_0$ \\
идея: индукция \\
Дифф. $k$ п. опр. по инд.:
\begin{equation*}
    d^k f(x_0) dx = d(d^{k-1} f(x) dx) \bigg |_{x=x_0} dx
\end{equation*}
идея: индукция \\
Лемма 1. Ф-ия $f$ $k$  раз дифф. в т.$x_0$ $\Rightarrow$
\begin{equation*}
    d^k f(x_0) dx = \sum_{i1=1}^n ... \sum_{ik=1}^n \frac{\partial ^k f(x_0)}{\partial x^{ik} ... \partial x^{i1}} dx^{ik} ... dx^{i1}
\end{equation*}
\subsection{Формула Тейлора для функций нескольких переменных с остаточным членом в форме Лагранжа и Пеано}
идея: фикс.$x$ и расс.: \\
\begin{equation*}
    \phi (t) = f(x_0+ \Delta x t)
\end{equation*}
теорема о сложной функции + формула Тейлора Лагранжа для одной переменной \\
Теорема 1. Ф-ия $f$ $(m+1)$ раз дифф. в окр.$x_0$. Тогда
\begin{equation*}
    \forall x \in U_\delta(x_0) \hookrightarrow
\end{equation*}
\begin{equation*}
    f(x) = f(x_0) + \sum_{k=1}^{m} \frac{d^k f(x_0)}{k!} \Delta x + \frac{d^{m+1} f(x_0+ \theta \Delta x)}{(m+1)!} \Delta x
\end{equation*}
\begin{equation*}
    \Delta x = x - x_0, \; \theta \in (0, 1)
\end{equation*}


\section{Теория меры}

\subsection{Определение измеримости по Лебегу множества в n–мерном евклидовом пространстве}
Мн-во $X \subset \mathbb R^n$ изм. по Лебегу $\Leftrightarrow$ явл. объед. сч.числа кон.изм.мн-в \\
Мн-во $X \subset \mathbb R^n$ кон.изм. $\Leftrightarrow$
\begin{equation*}
    \exists \{ X_k \}: \; X_k \; - \; cell \; set \; and \; X_k \overset{\mu}{\rightarrow} X
\end{equation*}
\subsection{Критерий измеримости}
\subsection{Измеримость объединения, пересечения и разности измеримых множеств}
идея: последовательности, для клеточных, переход к пределу \\
Лемма 3. Пусть мн-ва $X,Y \subset \mathbb R^n$ кон.изм. Тогда
\begin{equation*}
    X \cup Y, X \cap Y, X \setminus Y
\end{equation*}
кон.изм и
\begin{equation*}
    \mu (X \cup Y) + \mu (X \cap Y) = \mu (X) + \mu (Y)
\end{equation*}
\subsection{Счетная аддитивность меры Лебега}
идея: полуадд. - по опр., шаг 1 для кон.изм.\\
по лемме пред. каждое как дизъ. об., для которых уже док. \\
Пусть мн-во $X \subset \mathbb R^n$
\begin{equation*}
    X = \bigsqcup_{k=1}^{\infty} X_k, \; X_k - \; measurable \; set
\end{equation*}
Тогда Х изм. и
\begin{equation*}
    \mu (X) = \sum_{k=1}^\infty \mu ( X_k )
\end{equation*}
\subsection{Измеримость и мера цилиндра в (n+1) – мерном пространстве}


\section{Интеграл}
\subsection{Определенный интеграл Римана}
Опр.
\begin{equation*}
    J = \int_a^b f(x) dx \Leftrightarrow
\end{equation*}
\begin{equation*}
    \underset{l(T) \rightarrow 0}{lim} s = \underset{l(T) \rightarrow 0}{lim} S = J \Leftrightarrow
\end{equation*}
\begin{equation*}
    \forall \epsilon >0 \; \exists \delta > 0: \; \forall T: \; l(T)<\delta \hookrightarrow |s-J|<\epsilon \; and \; |S-J|<\epsilon
\end{equation*}
\subsection{Верхние и нижние суммы Дарбу, их свойства}
Опр. Пусть $f:[a,b] \rightarrow \mathbb R$ и зад. $T = \{ x_i \}_{i=0}^I$ отр. $[a,b]$ \\
Пусть $ m_i = \underset{x \in [x_{i-1}, x_i]}{inf} f(x) $, $ M_i = \underset{x \in [x_{i-1}, x_i]}{sup} f(x) $ \\
Тогда
\begin{equation*}
    s = \sum^I (x_i - x_{i-1})m_i
\end{equation*}
\begin{equation*}
    S = \sum^I (x_i - x_{i-1})M_i
\end{equation*}
наз. соотв. н.суммой Д. и в.суммой Д.
\subsection{Критерий интегрируемости}
идея: из опр. инт. берем $\frac{\epsilon}{2}$ \\
Теорема 1. $f:[a,b]\rightarrow\mathbb R$ инт. на $[a,b]$ $\Leftrightarrow$
\begin{equation*}
    \underset{l(T) \rightarrow 0}{lim} \Delta = 0 \Leftrightarrow
\end{equation*}
\begin{equation*}
    \forall \epsilon > 0 \; \exists \delta > 0: \; \forall T: \; l(T) \leq \delta \hookrightarrow \Delta \leq \epsilon
\end{equation*}
\subsection{Интегрируемость непрерывной функции, монотонной функции, ограниченной функции с конечным числом точек разрыва}
идея: т.К., модуль непр. стр. к нулю, огр. двумя \\
Теорема 1. $f(x)$ непр. на $[a,b]$ $\Rightarrow$ инт-ма на нем \\
\\
идея: рассм. один отр., по т.4 ф-я совп. за искл. кон. числа т. \\
Теорема 2. $f(x)$ кус. непр. на $[a,b]$ $\Rightarrow$ инт-ма на нем \\
\\
идея: кр.инт., $w_i = \underset{x \in [x_{i-1}, x_i]}{sup} |f(x'-x'')| = f(x_{i-1}) - f(x_{i}) $ \\
Теорема 3. $f(x)$ мон. на $[a,b]$ $\Rightarrow$ инт-ма на нем \\
\subsection{Аддитивность интеграла по отрезкам, линейность интеграла, интегрируемость произведения функций, интегрируемость модуля интегрируемой функции, интегрирование неравенств, теорема о среднем.}
идея: огр. на $[a,c]$,  $|S(T)-S(T_1)-S(T_2)| \leq 2M \cdot l(T) \rightarrow 0$ \\
Теорема 5. $f(x)$ инт-ма на $[a,b]$ и $[b,c]$ \\
Тогда $f(x)$ инт-ма на $[a,c]$ и
\begin{equation*}
    \int_a^c f(x) dx = \int_a^b f(x) dx + \int_b^c f(x) dx
\end{equation*}
\\
идея: лин. инт. суммы Р. \\
Теорема 1. $f(x), \; g(x)$ инт-мы на $[a,b]$ \\
То $\phi (x) = \alpha f(x) + \beta g(x) $ инт-ма на $[a,b] \; \forall \alpha, \; \beta \in \mathbb R$ и
\begin{equation*}
    \int_a^b (\alpha f(x) + \beta g(x)) dx = \alpha \int_a^b f(x) dx + \beta \int_a^b g(x) dx
\end{equation*}
\\
идея: $\sigma (f) \leq \sigma (g)$ пер. к пред. $J_f \leq J_g$ \\
Теорема 2. $f(x), \; g(x)$ инт-мы на $[a,b]$ и $\forall x \in [a,b]$ $\hookrightarrow$ $f(x) \leq g(x)$ $\Rightarrow$
\begin{equation*}
    \int_a^b f(x) dx \leq \int_a^b g(x) dx
\end{equation*}
\\
идея: \\
Теорема 3. $f(x)$ инт-ма на $[a,b]$ $\Rightarrow$ $|f(x)|$ инт-ма на $[a,b]$ и
\begin{equation*}
    |\int_a^b f(x) dx| \leq \int_a^b |f(x)| dx
\end{equation*}
\subsection{Свойства интеграла с переменным верхним пределом — непрерывность, дифференцируемость}
\subsection{Формула Ньютона–Лейбница}
идея: $F(a) = C, \; F(b) = \int_a^b f(x) dx + C \rightarrow F(b)-F(a) = \int_a^b f(x) dx $ \\
Следствие 2. $F(x)$ перв. $f(x)$ на $[a,b]$, $f(x)$ непр. $\Rightarrow$
\begin{equation*}
    \int_a^b f(x) dx = F(b) - F(a)
\end{equation*}
\subsection{Замена переменной и интегрирование по частям в определенном интеграле}


\section{Геометрические приложения интеграла}
\subsection{Геометрические приложения определенного интеграла — площадь криволинейной трапеции, объем тела вращения, длина кривой}
идея: разбить на супремум и инфимум прямоугольники, их суммы равны в.сумме Д. и н.сумме Д. \\
Теорема 1. Пусть $f:[a,b] \rightarrow \mathbb R$ инт. и неотр. на $[a,b]$ \\
Тогда кр.тр. 
\begin{equation*}
    E = \{ (x,y): a \leq x \leq b, \; 0 \leq y \leq f(x) \}
\end{equation*}
явл. изм. мн. и
\begin{equation*}
    \mu (E) = \int_a^b f(x) dx
\end{equation*}
идея: разбить на супремум и инфимум цилиндры, их суммы равны в.сумме Д. и н.сумме Д. \\
Теорема 2. Пусть $f:[a,b]\rightarrow\mathbb R$ инт. и неотр. на $[a,b]$ \\
Тогда тело вр.
\begin{equation*}
    G = \{ (x,y,z) \in \mathbb R^3: x \in [a,b], \sqrt{y^2+z^2} \leq f(x) \}
\end{equation*}
изм. и равно
\begin{equation*}
    \mu(G) = \pi \int_a^b f^2(x) dx
\end{equation*}
идея: $s' = |\overline{r}'|$ \\
Теорема 3. Если кр. $L = \{ \overline{r}(t): t \in [a,b] \}$ непр. дифф., то
\begin{equation*}
    |L| = \int_a^b |\frac{d\overline{r}}{dt}| dt
\end{equation*}
\subsection{Вычисление площади поверхности вращения (без
доказательства)}
Теорема 2. Пусть $f(x): [a,b] \rightarrow \mathbb R$ -- неотр., непр.дифф. \\
Тогда пл.пов.вр. сущ. и равна
\begin{equation*}
    S = 2\pi \int_a^b f(x) \sqrt{1+(\frac{df}{dx})^2}dx
\end{equation*}


\section{Криволинейный интеграл I и II рода}
\subsection{Криволинейный интеграл первого рода}
Определение. Пусть кр. $L = \{ \overline{r}, \; t \in [a,b] \} \subset \mathbb R^n$ зад. непр.в-р-ф-ей $\overline{r}$, пр. кот. кус.непр. на отр. $[a,b]$. Пусть на мн-ве $L$ зад. непр.скал.ф-я $f(\overline{r}(t))$ \\
Крив.инт. 1 рода ф-ии $f$ по кр. $L$ наз. \\
\begin{equation*}
    \underset{L}{\int} fds = \int_a^b f(\overline{r}(t))|r'(t)|dt
\end{equation*}
идея: два аргумента, функция из аргумента 1 в аргумент 2 (непр., возр., непр.дифф.), теорема о производной сл.ф-ии \\
Теорема 1. Крив. инт. 1 рода не зав. от параметризации. \\
идея: замена $a \rightarrow -b, \; b \rightarrow -a, \; t \rightarrow -t$ \\
Лемма 1. При изм. ориент. кр.инт. 1 рода не изм.
\subsection{Криволинейный интеграл второго рода}
Определение. Пусть кр. $L = \{ \overline{r}, \; t \in [a,b] \} \subset \mathbb R^n$ зад. непр.в-р-ф-ей $\overline{r}$, пр. кот. кус.непр. на отр. $[a,b]$. Пусть на мн-ве $L$ зад. непр.$n$-м.в-р-ф-я $\overline{F}(\overline{r}(t))$ \\
Крив.инт. 2 рода ф-ии $\overline F$ по кр. $L$ наз. \\
\begin{equation*}
    \underset{L}{\int} \overline F d \overline r = \int_a^b \overline F \overline r' dt
\end{equation*}
идея: ан-но крив. инт. 1 рода \\
Теорема 2. Крив. инт. 2 рода не зав. от параметризации. \\
идея: ан-но крив. инт. 1 рода \\
Лемма 2. При изм. ор. кр. крив. инт. 2 рода меняет знак.


\section{Несобственный интеграл}
\subsection{Несобственный интеграл}
Определение. Пусть $f:[a,\infty) \rightarrow \mathbb R \; and \; \forall b>a$ $f$ инт-ма на $[a,b]$ \\
Тогда несоб. инт. наз.
\begin{equation*}
    \int_a^\infty f(x) dx = \underset{b \rightarrow \infty}{lim} \int_a^b f(x) dx
\end{equation*}
\subsection{Критерий Коши сходимости интеграла}
идея: пред. к $b$ по кр.К. сущ. пред. ф-ии \\
Теорема 1. Пусть $f$ инт-ма на $\forall[c,d] \subset [a,b)$ \\
Тогда $\int_a^b f(x) dx$ сх. $\Leftrightarrow$
\begin{equation*}
    \forall \epsilon > 0 \; \exists \xi \in (a,b): \; \forall b_1,b_2 \in (\xi, b) \hookrightarrow |\int_{b1}^{b2}f(x)dx|<\epsilon
\end{equation*}
\subsection{Интегралы от знакопостоянных функций, признак сравнения сходимости}
идея: супремумы отрезков меньше или равны, соотв. док. интегралы меньше/больше бесконечности по транзитивности \\
Теорема 2. Пусть $f \; and \; g$ инт-мы на $\forall [c,d] \subset [a,b)$ и $\forall x \in [a,b) \hookrightarrow 0 \leq f(x) \leq g(x)$ \\
Тогда \\
а. из сх. $\int_a^b g dx$ $\hookrightarrow$ сх. $\int_a^b f dx$ \\
б. из расх. $\int_a^b f dx$ $\hookrightarrow$ расх. $\int_a^b g dx$
\subsection{Интегралы от знакопеременных функций, абсолютная и условная сходимость}
Определение. $\int_a^b f(x) dx$ сх. абс. $\Leftrightarrow$ $\int_a^b |f(x)| dx$ сх. \\
Определение.  $\int_a^b f(x) dx$ не сх.абс., но сх. $\Leftrightarrow$ $\int_a^b f(x) dx$ сх.усл. \\
идея: кр.К. сх. инт., нер-во с модулем \\
Теорема 2. Пусть $f$ инт-ма на $\forall [c,d] \subset [a,b)$ \\
$\int_a^b f(x) dx$ сх.абс. $\Rightarrow$ $\int_a^b f(x) dx$ сх.
\subsection{Признаки Дирихле и Абеля сходимости интегралов}
идея: по частям, $F(x)$ огр., по пр.ср., $g(x) \rightarrow 0$, не влияет \\
Теорема 3. Пусть $f(x)$ непр., а $g(x)$ непр.дифф на $[a,b), \; b \in \overline{\mathbb R}$ \\
Пусть \\
1. $F(x)$ огр. на $[a,b)$ \\
2. $\underset{x \rightarrow b-0}{lim} g(x) = 0$ \\
3. $\forall x \in [a,b) \; g'(x) \leq 0$
Тогда \\
\begin{equation*}
    \int_a^b f(x) \cdot g(x) dx
\end{equation*}
сх. \\

\section{Числовые ряды}
\subsection{Числовые ряды}
Определение. Пусть $\{ a_k \}_{k=1}^{\infty}$ -- числ. посл. \\
Сумма ряда
\begin{equation*}
    \sum^\infty a_k = \underset{n \rightarrow \infty}{lim} \sum^n a_k
\end{equation*}
Ряд наз. сх., если $\overset{\infty}{\sum} a_k \in \mathbb R$, в пр. сл. расх.
\subsection{Критерий Коши сходимости ряда}
идея: кр.К. для посл. \\
Теорема 1.
\begin{equation*}
    \overset{\infty}{\sum} a_k \in \mathbb R \Leftrightarrow
\end{equation*}
\begin{equation*}
    \forall \epsilon > 0 \; \exists N \in \mathbb N: \; \forall n \geq N \; \forall p \in \mathbb N \hookrightarrow |\sum_{n+1}^{n+p} a_k| < \epsilon
\end{equation*}
\subsection{Знакопостоянные ряды: признак сравнения сходимости, признаки Даламбера и Коши, интегральный признак}
идея: по т1 расс. супремумы и исп. транз. \\
Теорема 2.
\begin{equation*}
    0 \leq a_k \leq b_k \; \forall k \in \mathbb N \; \Rightarrow
\end{equation*}
а. из сх. $\overset{\infty}{\sum} b_k$ $\hookrightarrow$ сх. $\overset{\infty}{\sum} a_k$ \\
б. из расх. $\overset{\infty}{\sum} a_k$ $\hookrightarrow$ расх. $\overset{\infty}{\sum} b_k$ \\
идея: ср-ть с геом. прог. \\
Теорема 5.
\begin{equation*}
    a_k>0 \; \forall k \in \mathbb N
\end{equation*}
Тогда
\begin{equation*}
    a. \; if \; \exists k_0 \in \mathbb N \; and \; q \in (0,1): \frac{a_{k+1}}{a_k} \leq q \; \forall k \geq k_0 \; then \; \overset{\infty}{\sum} a_k \; conv.
\end{equation*}
\begin{equation*}
    b. \; if \; \exists k_0 \in \mathbb N: \frac{a_{k+1}}{a_k} \geq 1 \; \forall k \geq k_0 \; then \; \overset{\infty}{\sum} a_k \; disconv.
\end{equation*}
идея: возвести в ст., ср-ть с геом. прог. \\
Теорема 6.
\begin{equation*}
    a_k>0 \; \forall k \in \mathbb N
\end{equation*}
\begin{equation*}
    a. \; if \; \exists k_0 \in \mathbb N \; and \; q \in (0,1): \sqrt[k]{a_k} \leq q \; \forall k \geq k_0 \; then \; \overset{\infty}{\sum} a_k \; conv.
\end{equation*}
\begin{equation*}
    b. \; if \; \exists k_0 \in \mathbb N: \sqrt[k]{a_k} \geq 1 \; \forall k \geq k_0 \; then \; \overset{\infty}{\sum} a_k \; disconv.
\end{equation*}
идея: проинт. и просумм. $f(k+1) \leq f(x) \leq f(k)$, супр. \\
Теорема 4. $f(x)$ зад. и мон. на $[1, \infty)$ \\
Тогда $\overset{\infty}{\sum} f(k)$ и $\int_1^\infty f(x)dx$ сх. или расх. одн-но \\
\subsection{Знакопеременные ряды, абсолютная и условная сходимость, признаки Лейбница, Дирихле и Абеля}
Определение. $\overset{\infty}{\sum} a_k$ наз. абс. сх., если сх. $\overset{\infty}{\sum} |a_k|$ \\
$\overset{\infty}{\sum} a_k$ наз. усл. сх., если $\overset{\infty}{\sum} a_k$ сх., а $\overset{\infty}{\sum} |a_k|$ расх. \\
идея: кр.К. для рядов \\
Теорема 2. Ряд сх. абс. $\Rightarrow$ сх. \\
идея: пр-ие Абеля $A_n = \overset{n}{\sum} a_k, \; A_0 = 0 $ \\
Теорема 3.
\begin{equation*}
    \exists C \in \mathbb R: \forall n \in \mathbb N \hookrightarrow |\overset{n}{\sum} a_k| \leq C \; and \; b_k \overset{mon}{\rightarrow} 0 \; as \; k \rightarrow \infty
\end{equation*}
Тогда $\overset{\infty}{\sum} a_k b_k$ сх.
\subsection{Независимость суммы абсолютно сходящегося ряда от порядка слагаемых}
\subsection{Теорема Римана о перестановках
членов условно сходящегося ряда (без доказательства)}
идея: когда б., добирать отр., когда м., добирать пол. (разд. ряд на пол. и отр.) \\
Теорема 2. $\overset{\infty}{\sum} a_k$ сх. усл., то $\forall x \in \mathbb R$ можно пер. чл. $\overset{\infty}{\sum} a_k$ так: $\overset{\infty}{\sum} a_{kj} = x$
\subsection{Произведение абсолютно сходящихся рядов}


\section{Функциональные последовательности и ряды}
\subsection{Равномерная сходимость функциональных последовательностей и рядов}
Опр. $\{ f_n (x) \}$ сх. к $f(x)$ на мн.$X$ $\Leftrightarrow$
\begin{equation*}
    f_n (x) \underset{X}{\rightrightarrows} f(x) \Leftrightarrow
\end{equation*}
\begin{equation*}
    \forall \epsilon > 0 \; \exists N \in \mathbb N: \; \forall n \geq N \; \forall x \in X \; \hookrightarrow |f_n (x) - f(x)| \leq \epsilon
\end{equation*}
Для ф. р. опр. ан-ое, расс. посл. ч. сумм
\subsection{Критерий Коши равномерной сходимости}
идея: $\Rightarrow$ в опр. вз. $\epsilon/2$, $\Leftarrow$ в опр. вз. $p \rightarrow \infty$ \\
Теорема 2.
\begin{equation*}
    f_n (x) \underset{X}{\rightrightarrows} f(x) \Leftrightarrow
\end{equation*}
\begin{equation*}
    \forall \epsilon>0 \; \exists N \in \mathbb N: \; \forall n \geq N \; \forall p \in \mathbb N \; x \in X \hookrightarrow |f_n (x)-f_{n+p}(x)| \leq \epsilon
\end{equation*}
Для ф. р. опр. ан-ое, расс. посл. ч. сумм
\subsection{Непрерывность суммы равномерно сходящегося ряда из непрерывных функций}
идея: $|f_N(x) - f(x)| \leq \epsilon/4$, $|f_N(x)-f(x_0)| \leq \epsilon/2$ $\Rightarrow$ $|f(x)-f(x_0)| \leq \epsilon$ \\
Теорема 1. $f_n(x)$ непр. на $X$ и
\begin{equation*}
    f_n (x) \underset{X}{\rightrightarrows} f(x)
\end{equation*}
То $f(x)$ непр. на $X$
\subsection{Почленное интегрирование и дифференцирование функциональных последовательностей и рядов}
\subsection{Признаки Вейерштрасса, Дирихле и Абеля равномерной сходимости функциональных рядов}


\section{Степенные ряды}
\subsection{Степенные ряды с комплексными членами}
Опр. $\{ c_k \}$, $c_k, w_0 \in \mathbb C$
\begin{equation*}
    \sum_{k=0}^\infty c_k (w-w_0)^k
\end{equation*}
наз. ст. р.
\subsection{Первая теорема Абеля}
идея: по св-ву рад. сх. \\
Теорема 2. $\sum_{k=0}^\infty c_k z^k$ сх. в т.$z_0$ \\
Тогда
\begin{equation*}
    \forall z \in \mathbb C: |z| < |z_0|
\end{equation*}
исх. р. сх. абс.
\subsection{Круг и радиус сходимости}
Опр. Рад. сх. ст. р. $\sum_{k=0}^\infty c_k z^k$ наз.
\begin{equation*}
    R \in [0, \infty]: \; \frac{1}{R} = \underset{k \rightarrow \infty}{\overline{\lim}} \sqrt[k]{|c_k|}
\end{equation*}
Кр. с ц. в т.$(0,0)$ и рад. $R$ наз. кр. сх. ст. р. $\sum_{k=0}^\infty c_k z^k$
\subsection{Характер сходимости степенного ряда в круге сходимости}
идея: ср. $|c_k z^k| \leq |c_k| r^k$, сх. по св-ву рад., по пр.В.\\
Теорема 3. $R>0$ ст. р. $\sum_{k=0}^\infty c_k z^k$ $\Rightarrow$
\begin{equation*}
    \forall r \in (0, R) \; series \; \sum_{k=0}^\infty c_k z^k \; converges \; uniform \; in \; Z = \{ z \in \mathbb C: |z| \leq r \}
\end{equation*}
\subsection{Непрерывность суммы степенного ряда в круге сходимости}
\subsection{Формула Коши–Адамара}
\begin{equation*}
    \frac{1}{R} = \underset{k \rightarrow \infty}{\overline{\lim}} \sqrt[k]{|c_k|}
\end{equation*}
\subsection{Сохранение радиуса сходимости степенного ряда при формальном дифференцировании и интегрировании ряда}
идея: ф.К.А., из дифф. след. инт. \\
Теорема 5. Рад. ст. р. $\sum_{k=0}^\infty c_k k z^{k-1}$ и $\sum_{k=0}^\infty \frac{c_k}{k+1} z^{k+1}$ совп. с рад. $\sum_{k=0}^\infty c_k z^k$
\subsection{Вторая теорема Абеля}
идея: пр.А. \\
Теорема 4. Пусть $\sum_{k=0}^\infty c_k z^k$ сх. в $z_1 \in \mathbb C$ \\
Тогда $\sum_{k=0}^\infty c_k z^k$ сх. равн. на $[0, z_1] = \{ tz_1: t \in [0,1] \}$
\section{Ряд Тейлора}
\subsection{Степенные ряды с действительными членами}
\subsection{Бесконечная дифференцируемость суммы степенного ряда на интервале сходимости}
\subsection{Единственность представления функции степенным рядом}
\subsection{Достаточные условия разложимости бесконечно дифференцируемой функции в степенной ряд}
идея: ф.Т. по Л., $M \frac{\delta^{n+1}}{(n+1)!} \rightarrow 0 \; as \; n \rightarrow \infty$ \\
Теорема 1. $\exists \delta > 0: \; f$ беск. дифф. в $U_\delta (x_0)$ и 
\begin{equation*}
    \exists M > 0: \; \forall n \in \mathbb N \; \forall x \in U_\delta (x_0) \hookrightarrow |f^{(n)}(x)| \leq M \; \Rightarrow
\end{equation*}
$f$ рег. в т.$x_0$ и
\begin{equation*}
    \forall x \in U_\delta (x_0) \hookrightarrow f(x) = \sum_{k=0}^\infty \frac{f^{(k)} (x_0)}{k!} (x-x_0)^k
\end{equation*}
\subsection{Ряд Тейлора}
Опр. $f(x)$ -- беск. дифф. в т.$x_0$ \\
Тогда
\begin{equation*}
    \sum_{k=0}^\infty \frac{f^{(k)} (x_0)}{k!} (x-x_0)^k
\end{equation*}
наз. р.Т. ф-ии $f(x)$ в т.$x_0$
\subsection{Формула Тейлора с остаточным членом в интегральной форме}
\subsection{Пример бесконечно дифференцируемой функции, не разлагающейся в степенной ряд}
идея: все пр. в нуле равны нулю \\
Бесконечно дифференцируемая функция, не разлагающаяся в степенной ряд
\begin{equation*}
    f(x) = 
    \begin{cases}
    \exp (-1/x^2) & x \neq 0 \\
    0 & x=0
    \end{cases}
\end{equation*}

\subsection{Разложение в ряд Тейлора основных элементарных функций}
1. $e^x$ беск. дифф., $|(e^x)^{(n)}| \leq e^\delta \; \forall x \in U_\delta (0) \; \forall n \in \mathbb N \Rightarrow$
\begin{equation*}
    e^x = \sum_{k=0}^\infty \frac{x^k}{k!}
\end{equation*}
\subsection{Разложение в степенной ряд комплекснозначной экспоненты}
идея: док. $\exp (z_1) \cdot \exp (z_2) = \exp (z_1+z_2)$ через произ. р. и б.Н. $k_1 = k$, $k_2 = m-k$ \\
Теорема 2.
\begin{equation*}
    \forall z \in \mathbb C \hookrightarrow \exp(z) = \sum_{k=0}^\infty \frac{z^k}{k!}
\end{equation*}
\end{document}
